\documentclass[11pt, a4paper]{article}

% If you can't see cyrillic letters in R-studio choose
% File-Reopen with encoding
% utf8 is the preferred encoding


%%%%%%%%%%%%%%%%%%%%%%%  Загрузка пакетов  %%%%%%%%%%%%%%%%%%%%%%%%%%%%%%%%%%
% кусок от урсса
%\usepackage[60x90,headers,11pt]{format}

%\textheight=494pt%
%\textwidth=322pt%
%
%\oddsidemargin=0pt%
%\evensidemargin=0pt
%\topmargin=-1pt \headsep=14pt \headheight=22pt \voffset=-28pt
%\hoffset=-50pt


\clubpenalty=10000
\widowpenalty=10000

%\overfullrule=5pt
%\hfuzz=1.5mm
%\baselineskip=12pt plus 0.18pt minus 0.1pt


%\pagestyle{headings}
% конец куска от урсса




% специальная версия для knitr'а. Исключает graphicx

%\usepackage{showkeys} % показывать метки в готовом pdf

\usepackage{etex} % расширение классического tex
% в частности позволяет подгружать гораздо больше пакетов, чем мы и займёмся далее

%\usepackage{mathtext} % русские буквы в формулах? (и без неё работает)
% Например, $x_{\text{один}}$

%\usepackage{cmap} % для поиска русских слов в pdf --- теперь работает без этого
% а с cmap не работает печать на принтер ;)
\usepackage{verbatim} % для многострочных комментариев
\usepackage{makeidx} % для создания предметных указателей
\usepackage[X2, T2A]{fontenc}
\usepackage[utf8]{inputenc} % задание utf8 кодировки исходного tex файла
\usepackage{setspace}
\usepackage{amsmath, amsfonts, amssymb, amsthm}
\usepackage{mathrsfs} % sudo yum install texlive-rsfs
\usepackage{dsfont} % sudo yum install texlive-doublestroke
\usepackage{array, multicol, multirow, bigstrut} % sudo yum install texlive-multirow
\usepackage{indentfirst} % установка отступа в первом абзаце главы
\usepackage[russian]{babel} % выбор языка для документа
\usepackage{bm}
\usepackage{bbm} % шрифт с двойными буквами
%\usepackage[perpage]{footmisc}

\usepackage{dcolumn} % центрирование по разделителю для apsrtable

% создание гиперссылок в pdf
\usepackage[pdftex, unicode, colorlinks=true, urlcolor=blue, hyperindex, breaklinks]{hyperref}

% свешиваем пунктуацию
% теперь знаки пунктуации могут вылезать за правую границу текста, при этом текст выглядит ровнее
\usepackage{microtype}

\usepackage{textcomp}  % Чтобы в формулах можно было русские буквы писать через \text{}

% размер листа бумаги
%\usepackage[paperwidth=145mm,paperheight=215mm,
%height=182mm,width=113mm,top=20mm,includefoot]%{geometry}
\usepackage[paper=a4paper, top=15mm, bottom=13.5mm, left=16.5mm, right=13.5mm, includefoot]{geometry}

\usepackage{xcolor}

% \usepackage[pdftex]{graphicx} % для вставки графики, убрано, т.к. knitr похоже сам добавляет

\usepackage{float, longtable}
\usepackage{soulutf8}

\usepackage{enumitem} % дополнительные плюшки для списков
%  например \begin{enumerate}[resume] позволяет продолжить нумерацию в новом списке

\usepackage{mathtools}
\usepackage{cancel,xspace} % sudo yum install texlive-cancel

% \usepackage{minted} % display program code with syntax highlighting
% требует установки pygments и python

\usepackage{numprint} % sudo yum install texlive-numprint
\npthousandsep{,}\npthousandthpartsep{}\npdecimalsign{.}

\usepackage{embedfile} % Чтобы код LaTeXа включился как приложение в PDF-файл

\usepackage{subfigure} % для создания нескольких рисунков внутри одного

\usepackage{tikz, pgfplots} % язык для рисования графики из latex'a
\usepackage{tikz-3dplot} % для 3d графиков
\usetikzlibrary{trees} % tikz-прибамбас для рисовки деревьев
\usepackage{tikz-qtree} % альтернативный tikz-прибамбас для рисовки деревьев
\usetikzlibrary{arrows} % tikz-прибамбас для рисовки стрелочек подлиннее

\usepackage{todonotes} % для вставки в документ заметок о том, что осталось сделать
% \todo{Здесь надо коэффициенты исправить}
% \missingfigure{Здесь будет Последний день Помпеи}
% \listoftodos --- печатает все поставленные \todo'шки


% более красивые таблицы
\usepackage{booktabs}
% заповеди из докупентации:
% 1. Не используйте вертикальные линни
% 2. Не используйте двойные линии
% 3. Единицы измерения - в шапку таблицы
% 4. Не сокращайте .1 вместо 0.1
% 5. Повторяющееся значение повторяйте, а не говорите "то же"



%\usepackage{asymptote} % пакет для рисовки графики, должен идти после graphics
% но мы переходим на tikz :)

%\usepackage{sagetex} % для интеграции с Sage (вероятно тоже должен идти после graphics)

% metapost создает упрощенные eps файлы, которые можно напрямую включать в pdf
% эта группа команд декларирует, что файлы будут этого упрощенного формата
% если metapost не используется, то этот блок не нужен
\usepackage{ifpdf} % для определения, запускается ли pdflatex или просто латех
\ifpdf
	\DeclareGraphicsRule{*}{mps}{*}{}
\fi
%%%%%%%%%%%%%%%%%%%%%%%%%%%%%%%%%%%%%%%%%%%%%%%%%%%%%%%%%%%%%%%%%%%%%%


%%%%%%%%%%%%%%%%%%%%%%%  Внедрение tex исходников в pdf файл  %%%%%%%%%%%%%%%%%%%%%%%%%%%%%%%%%%
\embedfile[desc={Main tex file}]{\jobname.tex} % Включение кода в выходной файл
\embedfile[desc={title_bor}]{title_bor_utf8_knitr.tex}

%%%%%%%%%%%%%%%%%%%%%%%%%%%%%%%%%%%%%%%%%%%%%%%%%%%%%%%%%%%%%%%%%%%%%%



%%%%%%%%%%%%%%%%%%%%%%%  ПАРАМЕТРЫ  %%%%%%%%%%%%%%%%%%%%%%%%%%%%%%%%%%
\setstretch{1}                          % Межстрочный интервал
\flushbottom                            % Эта команда заставляет LaTeX чуть растягивать строки, чтобы получить идеально прямоугольную страницу
\righthyphenmin=2                       % Разрешение переноса двух и более символов
%\pagestyle{plain}                       % Нумерация страниц снизу по центру.
%\widowpenalty=300                     % Небольшое наказание за вдовствующую строку (одна строка абзаца на этой странице, остальное --- на следующей)
%\clubpenalty=3000                     % Приличное наказание за сиротствующую строку (омерзительно висящая одинокая строка в начале страницы)
\setlength{\parindent}{1.5em}           % Красная строка.
%\captiondelim{. }
\setlength{\topsep}{0pt}
%%%%%%%%%%%%%%%%%%%%%%%%%%%%%%%%%%%%%%%%%%%%%%%%%%%%%%%%%%%%%%%%%%%%%%



%%%%%%%% Это окружение, которое выравнивает по центру без отступа, как у простого center
\newenvironment{center*}{%
  \setlength\topsep{0pt}
  \setlength\parskip{0pt}
  \begin{center}
}{%
  \end{center}
}
%%%%%%%%%%%%%%%%%%%%%%%%%%%%%%%%%%%%%%%%%%%%%%%%%%%%%%%%%%%%%%%%%%%%%%


%%%%%%%%%%%%%%%%%%%%%%%%%%% Правила переноса  слов
\hyphenation{ }
%%%%%%%%%%%%%%%%%%%%%%%%%%%%%%%%%%%%%%%%%%%%%%%%%%%%%%%%%%%%%%%%%%%%%%

\emergencystretch=2em


% DEFS
\def \mbf{\mathbf}
\def \msf{\mathsf}
\def \mbb{\mathbb}
\def \tbf{\textbf}
\def \tsf{\textsf}
\def \ttt{\texttt}
\def \tbb{\textbb}

\def \wh{\widehat}
\def \wt{\widetilde}
\def \ni{\noindent}
\def \ol{\overline}
\def \cd{\cdot}
\def \fr{\frac}
\def \bs{\backslash}
\def \lims{\limits}
\DeclareMathOperator{\dist}{dist}
\DeclareMathOperator{\VC}{VCdim}
\DeclareMathOperator{\card}{card}
\DeclareMathOperator{\sign}{sign}
\DeclareMathOperator{\sgn}{sign}
\DeclareMathOperator{\Tr}{\mbf{Tr}}
\DeclareMathOperator{\tr}{tr}


\def \xfs{(x_1,\ldots,x_{n-1})}
\DeclareMathOperator*{\argmin}{arg\,min}
\DeclareMathOperator*{\amn}{arg\,min}
\DeclareMathOperator*{\amx}{arg\,max}
\DeclareMathOperator{\trace}{tr}
\DeclareMathOperator{\rk}{rk}


\DeclareMathOperator{\Corr}{Corr}
\DeclareMathOperator{\sCorr}{sCorr}
\DeclareMathOperator{\sCov}{sCov}
\DeclareMathOperator{\sVar}{sVar}

\DeclareMathOperator{\Cov}{Cov}
\DeclareMathOperator{\Var}{Var}
\DeclareMathOperator{\corr}{Corr}
\DeclareMathOperator{\cov}{Cov}
\DeclareMathOperator{\var}{Var}
\DeclareMathOperator{\bin}{Bin}
\DeclareMathOperator{\Bin}{Bin}
\DeclareMathOperator{\rang}{rang}
\DeclareMathOperator*{\plim}{plim}
\DeclareMathOperator{\MSE}{MSE}

\providecommand{\iff}{\Leftrightarrow}
\providecommand{\hence}{\Rightarrow}

\def \ti{\tilde}
\def \wti{\widetilde}

\def \mA{\mathcal{A}}
\def \mB{\mathcal{B}}
\def \mC{\mathcal{C}}
\def \mE{\mathcal{E}}
\def \mF{\mathcal{F}}
\def \mH{\mathcal{H}}
\def \mL{\mathcal{L}}
\def \mN{\mathcal{N}}
\def \mU{\mathcal{U}}
\def \mV{\mathcal{V}}
\def \mW{\mathcal{W}}


\def \R{\mbb R}
\def \N{\mbb N}
\def \Z{\mbb Z}
\def \P{\mbb{P}}
\def \p{\mbb{P}}
\newcommand{\E}{\mathbb{E}}
\def \D{\msf{D}}
\def \I{\mbf{I}}

\def \QQ{\mbb Q}
\def \RR{\mbb R}
\def \NN{\mbb N}
\def \ZZ{\mbb Z}
\def \PP{\mbb P}


\def \a{\alpha}
\def \b{\beta}
\def \t{\tau}
\def \dt{\delta}
\newcommand{\e}{\varepsilon}
\def \ga{\gamma}
\def \kp{\varkappa}
\def \la{\lambda}
\def \sg{\sigma}
\def \sgm{\sigma}
\def \tt{\theta}
\def \ve{\varepsilon}
\def \Dt{\Delta}
\def \La{\Lambda}
\def \Sgm{\Sigma}
\def \Sg{\Sigma}
\def \Tt{\Theta}
\def \Om{\Omega}
\def \om{\omega}

%\newcommand{\p}{\partial}

\def \ni{\noindent}
\def \lq{\glqq}
\def \rq{\grqq}
\def \lbr{\linebreak}
\def \vsi{\vspace{0.1cm}}
\def \vsii{\vspace{0.2cm}}
\def \vsiii{\vspace{0.3cm}}
\def \vsiv{\vspace{0.4cm}}
\def \vsv{\vspace{0.5cm}}
\def \vsvi{\vspace{0.6cm}}
\def \vsvii{\vspace{0.7cm}}
\def \vsviii{\vspace{0.8cm}}
\def \vsix{\vspace{0.9cm}}
\def \VSI{\vspace{1cm}}
\def \VSII{\vspace{2cm}}
\def \VSIII{\vspace{3cm}}

\newcommand{\bls}[1]{\boldsymbol{#1}}
\newcommand{\bsA}{\boldsymbol{A}}
\newcommand{\bsH}{\boldsymbol{H}}
\newcommand{\bsI}{\boldsymbol{I}}
\newcommand{\bsP}{\boldsymbol{P}}
\newcommand{\bsR}{\boldsymbol{R}}
\newcommand{\bsS}{\boldsymbol{S}}
\newcommand{\bsX}{\boldsymbol{X}}
\newcommand{\bsY}{\boldsymbol{Y}}
\newcommand{\bsZ}{\boldsymbol{Z}}
\newcommand{\bse}{\boldsymbol{e}}
\newcommand{\bsq}{\boldsymbol{q}}
\newcommand{\bsy}{\boldsymbol{y}}
\newcommand{\bsbeta}{\boldsymbol{\beta}}
\newcommand{\fish}{\mathrm{F}}
\newcommand{\Fish}{\mathrm{F}}
\renewcommand{\phi}{\varphi}
\newcommand{\ind}{\mathds{1}}
\newcommand{\inds}[1]{\mathds{1}_{\{#1\}}}
\renewcommand{\to}{\rightarrow}
\newcommand{\sumin}{\sum\limits_{i=1}^n}
\newcommand{\ofbr}[1]{\bigl( \{ #1 \} \bigr)}     % Например, вероятность события. Большие круглые, нормальные фигурные скобки вокруг аргумента
\newcommand{\Ofbr}[1]{\Bigl( \bigl\{ #1 \bigr\} \Bigr)} % Например, вероятность события. Больше больших круглые, большие фигурные скобки вокруг аргумента
\newcommand{\oeq}{{}\textcircled{\raisebox{-0.4pt}{{}={}}}{}} % Равно в кружке
\newcommand{\og}{\textcircled{\raisebox{-0.4pt}{>}}}  % Знак больше в кружке

% вместо горизонтальной делаем косую черточку в нестрогих неравенствах
\renewcommand{\le}{\leqslant}
\renewcommand{\ge}{\geqslant}
\renewcommand{\leq}{\leqslant}
\renewcommand{\geq}{\geqslant}


\newcommand{\figb}[1]{\bigl\{ #1  \bigr\}} % большие фигурные скобки вокруг аргумента
\newcommand{\figB}[1]{\Bigl\{ #1  \Bigr\}} % Больше больших фигурные скобки вокруг аргумента
\newcommand{\parb}[1]{\bigl( #1  \bigr)}   % большие скобки вокруг аргумента
\newcommand{\parB}[1]{\Bigl( #1  \Bigr)}   % Больше больших круглые скобки вокруг аргумента
\newcommand{\parbb}[1]{\biggl( #1  \biggr)} % большие-большие круглые скобки вокруг аргумента
\newcommand{\br}[1]{\left( #1  \right)}    % круглые скобки, подгоняемые по размеру аргумента
\newcommand{\fbr}[1]{\left\{ #1  \right\}} % фигурные скобки, подгоняемые по размеру аргумента
\newcommand{\eqdef}{\mathrel{\stackrel{\rm def}=}} % знак равно по определению
\newcommand{\const}{\mathrm{const}}        % const прямым начертанием
\newcommand{\zdt}[1]{\textit{#1}}
\newcommand{\ENG}[1]{\foreignlanguage{british}{#1}}
\newcommand{\ENGs}{\selectlanguage{british}}
\newcommand{\RUSs}{\selectlanguage{russian}}
\newcommand{\iid}{\text{i.\hspace{1pt}i.\hspace{1pt}d.}}

\newdimen\theoremskip
\theoremskip=0pt
\newenvironment{note}{\par\vskip\theoremskip\textbf{Замечание.\xspace}}{\par\vskip\theoremskip}
\newenvironment{hint}{\par\vskip\theoremskip\textbf{Подсказка.\xspace}}{\par\vskip\theoremskip}
\newenvironment{ist}{\par\vskip\theoremskip Источник:\xspace}{\par\vskip\theoremskip}

\newcommand*{\tabvrulel}[1]{\multicolumn{1}{|c}{#1}}
\newcommand*{\tabvruler}[1]{\multicolumn{1}{c|}{#1}}

\newcommand{\II}{{\fontencoding{X2}\selectfont\CYRII}}   % I десятеричное (английская i неуместна)
\newcommand{\ii}{{\fontencoding{X2}\selectfont\cyrii}}   % i десятеричное
\newcommand{\EE}{{\fontencoding{X2}\selectfont\CYRYAT}}  % ЯТЬ
\newcommand{\ee}{{\fontencoding{X2}\selectfont\cyryat}}  % ять
\newcommand{\FF}{{\fontencoding{X2}\selectfont\CYROTLD}} % ФИТА
\newcommand{\ff}{{\fontencoding{X2}\selectfont\cyrotld}} % фита
\newcommand{\YY}{{\fontencoding{X2}\selectfont\CYRIZH}}  % ИЖИЦА
\newcommand{\yy}{{\fontencoding{X2}\selectfont\cyrizh}}  % ижица

%%%%%%%%%%%%%%%%%%%%% Определение разрядки разреженного текста и задание красивых многоточий
\sodef\so{}{.15em}{1em plus1em}{.3em plus.05em minus.05em}
\newcommand{\ldotst}{\so{...}}
\newcommand{\ldotsq}{\so{?\hbox{\hspace{-0.61ex}}..}}
\newcommand{\ldotse}{\so{!..}}
%%%%%%%%%%%%%%%%%%%%%%%%%%%%%%%%%%%%%%%%%%%%%%%%%%%%%%%%%%%%%%%%%%%%%%

%%%%%%%%%%%%%%%%%%%%%%%%%%%%% Команда для переноса символов бинарных операций
\def\hm#1{#1\nobreak\discretionary{}{\hbox{$#1$}}{}}
%%%%%%%%%%%%%%%%%%%%%%%%%%%%%%%%%%%%%%%%%%%%%%%%%%%%%%%%%%%%%%%%%%%%%%

%\setlist[enumerate,1]{label=\arabic*., ref=\arabic*, partopsep=0pt plus 2pt, topsep=0pt plus 1.5pt,itemsep=0pt plus .5pt,parsep=0pt plus .5pt}
%\setlist[itemize,1]{partopsep=0pt plus 2pt, topsep=0pt plus 1.5pt,itemsep=0pt plus .5pt,parsep=0pt plus .5pt}

% Эти парни затем, если вдруг не захочется управлять списками из-под уютненького enumitem
% или если будет жизненно важно, чтобы в списках были именно русские буквы.
%\setlength{\partopsep}{0pt plus 3pt}
%\setlength{\topsep}{0pt plus 2pt}
%\setlength{\itemsep}{0 plus 1pt}
%\setlength{\parsep}{0 plus 1pt}

%на всякий случай пока есть
%теоремы без нумерации и имени
%\newtheorem*{theor}{Теорема}

%"Определения","Замечания"
%и "Гипотезы" не нумеруются
%\newtheorem*{defin}{Определение}
%\newtheorem*{rem}{Замечание}
%\newtheorem*{conj}{Гипотеза}

%"Теоремы" и "Леммы" нумеруются
%по главам и согласованно м/у собой
%\newtheorem{theorem}{Теорема}
%\newtheorem{lemma}[theorem]{Лемма}

% Утверждения нумеруются по главам
% независимо от Лемм и Теорем
%\newtheorem{prop}{Утверждение}
%\newtheorem{cor}{Следствие}


\def \useR{$[$R$]$ }

%% эконометрические сокращения
\def \hb{\hat{\beta}}
\def \hs{\hat{s}}
\def \hy{\hat{y}}
\def \hY{\hat{Y}}
\def \he{\hat{\varepsilon}}
\def \v1{\vec{1}}
\def \e{\varepsilon}
\def \z{z}
\def \hVar{\widehat{\Var}}
\def \hCorr{\widehat{\Corr}}
\def \hCov{\widehat{\Cov}}
\DeclareMathOperator{\Lin}{Lin}
\DeclareMathOperator{\Linp}{Lin^{\perp}}


%% лаг
\renewcommand{\L}{\mathrm{L}}





\usepackage[bibencoding = auto, backend = biber,
sorting = none]{biblatex}

\addbibresource{metrics_pro.bib}

\def \RR{\mathbb{R}}
\def \cN{\mathcal{N}}
\def \htheta{\hat{\theta}}

\title{Заметки к семинарам по эконометрике}
\author{Винни-Пух}
\date{\today}


% делаем короче интервал в списках
\setlength{\itemsep}{0pt}
\setlength{\parskip}{0pt}
\setlength{\parsep}{0pt}


\DeclareMathOperator{\Med}{Med}


\usepackage{answers} % дележка условий и ответов

%\newtheorem{problem}{Задача}
%\numberwithin{problem}{section}

\Newassociation{sol}{solution}{solution_file}
% sol — имя окружения внутри задач
% solution — имя окружения внутри solution_file
% solution_file — имя файла в который будет идти запись решений
% можно изменить далее по ходу
\Opensolutionfile{solution_file}[all_solutions]
% в квадратных скобках фактическое имя файла


% магия для автоматических гиперссылок задача-решение
\newlist{myenum}{enumerate}{3}
% \newcounter{problem}[chapter] % нумерация задач внутри глав
\newcounter{problem}

\newenvironment{problem}%
{%
\refstepcounter{problem}%
%  hyperlink to solution
     \hypertarget{problem:{\thesection.\theproblem}}{} % нумерация внутри глав
     % \hypertarget{problem:{\theproblem}}{}
     \Writetofile{solution_file}{\protect\hypertarget{soln:\thesection.\theproblem}{}}
     %\Writetofile{solution_file}{\protect\hypertarget{soln:\theproblem}{}}
     \begin{myenum}[label=\bfseries\protect\hyperlink{soln:\thesection.\theproblem}{\thesection.\theproblem},ref=\thesection.\theproblem]
     % \begin{myenum}[label=\bfseries\protect\hyperlink{soln:\theproblem}{\theproblem},ref=\theproblem]
     \item%
    }%
    {%
    \end{myenum}}
% для гиперссылок обратно надо переопределять окружение
% это происходит непосредственно перед подключением файла с решениями





\begin{document}

% \maketitle % ставим сюда название, автора и время создания

\section{МНК — это\ldots}

Минитеория:

\begin{enumerate}
\item Истинная модель. Например, $y_i = \beta_1 + \beta_2 x_i + \beta_3 z_i + u_i$.
\item Формула для прогнозов. Например, $\hy_i = \hb_1 + \hb_2 x_i + \hb_3 z_i$.
\item Метод наименьших квадратов, $\sum (y_i - \hy_i)^2 \to \min$.
\end{enumerate}

Задачи:
\begin{problem}
Каждый день Маша ест конфеты и решает задачи по эконометрике. Пусть $x_i$ — количество решённых задач, а $y_i$ — количество съеденных конфет.

\begin{tabular}{cc}
\toprule
$x_i$ & $y_i$ \\
\midrule
1 & 1 \\
2 & 2 \\
2 & 4 \\
\bottomrule
\end{tabular}

\begin{enumerate}
\item Рассмотрим модель $y_i = \beta x_i + u_i$:
\begin{enumerate}
\item Найдите МНК-оценку $\hb$ для имеющихся трёх наблюдений.
\item Нарисуйте исходные точки и полученную прямую регрессии.
\item Выведите формулу для $\hb$ в общем виде для $n$ наблюдений.
\end{enumerate}

\item Рассмотрим модель $y_i = \beta_1 + \beta_2 x_i + u_i$:
\begin{enumerate}
\item Найдите МНК-оценки $\hb_1$ и $\hb_2$ для имеющихся трёх наблюдений.
\item Нарисуйте исходные точки и полученную прямую регрессии.
\item Выведите формулу для $\hb_2$ в общем виде для $n$ наблюдений.
\end{enumerate}

\end{enumerate}


\begin{sol}
\end{sol}
\end{problem}


\begin{problem}
Упростите выражения:
\begin{enumerate}
\item $n\bar x - \sum x_i$
\item $\sum (x_i - \bar x)\bar x$
\item $\sum (x_i - \bar x)\bar z$
\item $\sum (x_i - \bar x)^2 + n \bar{x}^2$
\end{enumerate}

\begin{sol}
Ответы: $0$, $0$, $0$, $\sum x_i^2$.
\end{sol}
\end{problem}


\begin{problem}
При помощи метода наименьших квадратов найдите оценку неизвестного параметра $\theta$ в следующих моделях:

\begin{enumerate}
\item $y_i = \theta + \theta x_i + \varepsilon_i$;
\item $y_i = 1 + \theta x_i + \e_i$;
\item $y_i = \theta / x_i + \e_i$;
\item $y_i = \theta x_i + (1-\theta)z_i+\e_i$.
\end{enumerate}

\begin{sol}
\begin{enumerate}
\item \(\htheta = \sum \left((y_i - z_i)(x_i - z_i) \right) / \sum \left(x_i - z_i\right)^2 \)
\end{enumerate}
\end{sol}
\end{problem}

\begin{problem}
Найдите МНК-оценки параметров $\alpha$ и $\beta$ в модели $y_i = \alpha + \beta y_i + \e_i$.


\begin{sol}
\(\hat{\alpha} = 0, \ \hb = 1 \)
\end{sol}
\end{problem}


\begin{problem}
Рассмотрите модели $y_i = \alpha + \beta (y_i + z_i) + \e_i$, $z_i = \gamma + \delta(y_i+z_i) + \e_i$.
\begin{enumerate}
\item Как связаны между собой $\hat{\alpha}$ и $\hat{\gamma}$?
\item Как связаны между собой $\hb$ и $\hat{\delta}$?
\end{enumerate}


\begin{sol} % 1.5.
Рассмотрим регрессию суммы $(y_i + z_i)$ на саму себя. Естественно, в ней
\[
\widehat{y_i + z_i} = 0 + 1 \cdot (y_i + z_i).
\]

Отсюда получаем, что $\hat{\alpha} + \hat{\gamma} = 0$ и $\hb + \hat{\delta} = 1$.
\end{sol}
\end{problem}




\begin{problem}
Как связаны МНК-оценки параметров $\alpha, \beta$ и $\gamma, \delta$ в моделях $y_i = \alpha + \beta x_i + \e_i$ и $z_i = \gamma + \delta x_i + \upsilon_i$, если $z_i = 2 y_i$?


\begin{sol}

Исходя из условия, нужно оценить методом МНК коэффициенты двух следующих моделей:
\[y_i = \alpha + \beta x_i + \e_i \]
\[y_i = \frac{\gamma}{2} + \frac{\delta}{2} x_i + \frac{1}{2} v_i \]

Заметим, что на минимизацию суммы квадратов остатков коэффициент \(1/2\) не влияет, следовательно:
\[\hat{\gamma} = 2\hat{\alpha}, \ \hat{\delta} = 2 \hb  \]

\end{sol}
\end{problem}


\begin{problem}
Для модели $y_i = \beta_1 x_i + \beta_2 z_i + \e_i$ решите условную задачу о наименьших квадратах:
\[
Q(\beta_1, \beta_2) := \sum_{i=1}^n (y_i - \hb_1 x_i - \hb_2 z_i)^2 \rightarrow \underset{\beta_1 + \beta_2 = 1}{\min}.
\]


\begin{sol}
Выпишем задачу:
\[
\begin{cases}
RSS = \sum\limits_{i=1}^{n}(y_i - \hb_1x_i - \hb_2z_i)^2 \rightarrow \min\limits_{\hb_1, \hb_2}\\
\hb_1 + \hb_2 = 1
\end{cases}
\]

Можем превратить ее в задачу минимизации функции одного аргумента:
\[
RSS =  \sum\limits_{i=1}^{n}(y_i - x_i - \hb_2(z_i-x_i))^2 \rightarrow \min_{\hb_2}
\]

Выпишем условия первого порядка:
\[
\frac{\partial RSS}{\partial \hb_2} = \sum\limits_{i=1}^{n}2(y_i-x_i-\hb_2(z_i-x_i))(x_i-z_i)=0
\]

Отсюда:
\[
\sum\limits_{i=1}^{n}(y_i-x_i)(x_i-z_i) + \hb_2\sum\limits_{i=1}^{n}(z_i-x_i)^2 = 0 \Rightarrow \hb_2 = \frac{\sum\limits_{i=1}^n (y_i-x_i)(z_i-x_i)}{\sum\limits_{i=1}^n (z_i-x_i)^2}
\]

А $\hb_1$ найдется из соотношения $\hb_1+\hb_2 = 1$.

\end{sol}
\end{problem}

\begin{problem}
Перед нами два золотых слитка и весы, производящие взвешивания с ошибками. Взвесив первый слиток, мы получили результат $300$ грамм, взвесив второй слиток — $200$ грамм, взвесив оба слитка — $400$ грамм. Оцените вес каждого слитка методом наименьших квадратов.

\begin{sol}
Обозначив вес первого слитка за \(\beta_1\), вес второго слитка за \(\beta_2\), а показания весов за \(y_i\), получим, что
\[y_1 = \beta_1 + \e_1, \ y_2 = \beta_2 + \e_2, \ y_3 = \beta_1 + \beta_2 + \e_3\]

Тогда
\[(300 - \beta_1)^2 + (200 - \beta_2)^2 + (400 - \beta_1 - \beta_2)^2 \rightarrow \min \limits_{\beta_1,\  \beta_2} \]
\[\hb_1 = \frac{800}{3}, \ \hb_2 = \frac{500}{3} \]
\end{sol}
\end{problem}


\begin{problem}
Аня и Настя утверждают, что лектор опоздал на 10 минут. Таня считает, что лектор опоздал на 3 минуты. С помощью МНК оцените, на сколько опоздал лектор.

\begin{sol}
Можем воспользоваться готовой формулой для регрессии на константу:
\[
\hb = \bar{y} = \frac{10+10+3}{3} = \frac{23}{3}
\]

(можно решить задачу $2(10-\beta)^2 + (3-\beta)^2\rightarrow \min$)

\end{sol}
\end{problem}

\begin{problem}
Есть двести наблюдений. Вовочка оценил модель $\hy=\hb_1+\hb_2 x$ по первой сотне наблюдений. Петечка оценил модель $\hy=\hat{\gamma}_1+\hat{\gamma}_2 x$ по второй сотне наблюдений. Машенька оценила модель $\hy=\hat{m}_1+\hat{m}_2 x$ по всем наблюдениям.
\begin{enumerate}
\item Возможно ли, что $\hb_2>0$, $\hat{\gamma}_2>0$, но $\hat{m}_2<0$?
\item Возможно ли, что $\hb_1>0$, $\hat{\gamma}_1>0$, но $\hat{m}_1<0$?
\item Возможно ли одновременное выполнение всех упомянутых условий?
\end{enumerate}

\begin{sol}
\end{sol}
\end{problem}


\begin{problem}
У эконометриста Вовочки есть переменная $1_f$, которая равна 1, если $i$-ый человек в выборке — женщина, и 0, если мужчина. Есть переменная $1_m$, которая равна 1, если $i$-ый человек в выборке — мужчина, и 0, если женщина. Вовочка попробовал оценить 4 регрессии.
\begin{enumerate}
\item $y$ на константу и $1_f$;
\item $y$ на константу и $1_m$;
\item $y$ на $1_f$ и $1_m$ без константы;
\item $y$ на константу, $1_f$ и $1_m$.
\end{enumerate}

\begin{enumerate}
\item Какой смысл будут иметь оцениваемые коэффициенты?
\item Как связаны между собой оценки коэффициентов этих регрессий?
\end{enumerate}


\begin{sol}
\end{sol}
\end{problem}


\begin{problem}
Эконометрист Вовочка оценил методом наименьших квадратов модель 1, $y=\b_1+\b_2 x+\b_3 z+\e$, а затем модель 2, $y=\b_1+\b_2 x+\b_3 z+\b_4 w+\e$. Сравните полученные $ESS$, $RSS$, $TSS$ и $R^2$.

\begin{sol}
\end{sol}
\end{problem}


\begin{problem}
Что происходит с $TSS$, $RSS$, $ESS$, $R^2$ при добавлении нового наблюдения? Если величина может изменяться только в одну сторону, то докажите это. Если возможны и рост, и падение, то приведите пример.


\begin{sol}
Пусть \(\bar{y}\) — средний \(y\) до добавления нового наблюдения, \(\bar{y}'\) — после добавления нового наблюдения. Будем считать, что изначально было \(n\) наблюдений. Заметим, что
\[\bar{y}' = \frac{(y_1 + \ldots + y_n) + y_{n+1}}{n + 1} = \frac{n \bar{y} + y_{n + 1}}{n + 1} = \frac{n}{n+ 1}\bar{y} + \frac{1}{n+1}y_{n+1}\]

Покажем, что \(TSS\) может только увеличится при добавлении нового наблюдения (остается неизменным при \(y_{n+1} = \bar{y}\)):
\[TSS'= \sum_{i = 1}^{n + 1} (y_i - \bar{y}')^2 = \sum_{i = 1}^{n} (y_i - \bar{y} + \bar{y} - \bar{y}')^2 + (y_{n + 1} - \bar{y}')^2 = \]
\[=\sum_{i = 1}^{n} (y_i - \bar{y})^2 + n(\bar{y} - \bar{y}')^2 + (y_{n + 1} - \bar{y}')^2  = TSS + \frac{n}{n+1} (y_{n+1} - \bar{y})^2\]

Следовательно, \(TSS' \geqslant TSS\).

Также сумма \(RSS\) может только вырасти или остаться постоянной при добавлении нового наблюдения. Действительно, новое $(n+1)$-ое слагаемое в сумме неотрицательно. А сумма $n$ слагаемых минимальна при старых коэффициентах, а не при новых.

\(ESS\) и \(R^2\) могут меняться в обе стороны. Например, рассмотрим ситуацию, где точки лежат симметрично относительно некоторой горизонтальной прямой. При этом $ESS=0$. Добавим наблюдение — $ESS$ вырастет, удалим наблюдение — $ESS$ вырастет.
\end{sol}
\end{problem}



\begin{problem}
Эконометресса Аглая подглядела, что у эконометрессы Жозефины получился $R^2$ равный $0.99$ по 300 наблюдениям. От чёрной зависти Аглая не может ни есть, ни спать.

\begin{enumerate}
\item Аглая добавила в набор данных Жозефины ещё 300 наблюдений с такими же регрессорами, но противоположными по знаку игреками, чем были у Жозефины. Как изменится $R^2$?
\item Жозефина заметила, что Аглая добавила 300 наблюдений и вычеркнула их, вернув в набор данных в исходное состояние. Хитрая Аглая решила тогда добавить всего одно наблюдение так, чтобы $R^2$ упал до нуля. Удастся ли ей это сделать?
\end{enumerate}


\begin{sol}
\begin{enumerate}
\item $R^2$ упал до нуля.
\item Да, можно. Если добавить точку далеко слева внизу от исходного набора данных, то наклон линии регрессии будет положительный. Если далеко справа внизу, то отрицательный. Будем двигать точку так, чтобы поймать нулевой наклон прямой. Получим $ESS=0$.
\end{enumerate}
\end{sol}
\end{problem}


\begin{problem}
На работе Феофан построил парную регрессию по трём наблюдениям и посчитал прогнозы $\hat{y}_i$. Придя домой он отчасти вспомнил результаты:

\begin{tabular}{rr}
\toprule
$y_i$ & $\hy_i$ \\
\midrule
$0$ & $1$ \\
$6$ & ? \\
$6$ & ? \\
\bottomrule
\end{tabular}

Поднапрягшись, Феофан вспомнил, что третий прогноз был больше второго. Помогите Феофану восстановить пропущенные значения.


\begin{sol}
На две неизвестных $a$ и $b$ нужно два уравнения. Эти два уравнения — ортогональность вектора остатков плоскости регрессоров. А именно:

\[
\begin{cases}
\sum_i (y_i - \hy_i) = 0 \\
\sum_i (y_i - \hy_i) \hy_i = 0 \\
\end{cases}
\]

В нашем случае

\[
\begin{cases}
-1 +(6-a) + (6-b) = 0 \\
-1 + (6 - a)a + (6-b)b = 0 \\
\end{cases}
\]

Решаем квадратное уравнение и получаем два решения: $a=4$ и $a=7$. Итого: $a=4$, $b=7$.
\end{sol}
\end{problem}


\section{МНК в матрицах!}

Минитеория.

Дифференциал для матриц подчиняется правилам:

\begin{enumerate}
\item $da = 0$, $dA = 0$;
\item $d(RS) = dR \dot S + R \cdot dS$
\end{enumerate}

\begin{problem}
Пусть $t$ — скалярная переменная, $r$, $s$ — векторные переменные, $R$, $S$ — матричные переменные. Кроме того, $a$, $b$ — векторы констант, $A$, $B$ — матрицы констант.

Применив базовые правила дифференцирования найдите:
\begin{enumerate}
\item $d(ARB)$;
\item $d(r'r)$;
\item $d(r'Ar)$;
\item $d(R^{-1})$, воспользовавшись тем, что $R^{-1} \cdot R = I$;
\item $d \cos(r'r)$;
\item $d(r'Ar/r'r)$.
\end{enumerate}


\begin{sol}
\end{sol}
\end{problem}


\begin{problem}
В методе наименьших квадратов минимизируется функция
\[
Q(\hb) = (y - X\hb)'(y - X\hb).
\]

\begin{enumerate}
\item Найдите $dQ(\hb)$ и $d^2Q(\hb)$;
\item Выпишите условия первого порядка для задачи МНК;
\item Выразите $\hb$ предполагая, что $X'X$ обратима.
\end{enumerate}


\begin{sol}
\end{sol}
\end{problem}

\begin{problem}
В методе LASSO минимизируется функция
\[
Q(\hb) = (y - X\hb)'(y - X\hb) + \lambda hb' hb,
\]
где $\lambda$ — положительный параметр, штрафующий функцию за слишком большие значения $\hb$.

\begin{enumerate}
\item Найдите $dQ$;
\item Выпишите условия первого порядка для задачи LASSO;
\item Выразите $\hb$.
\end{enumerate}

\begin{sol}
\end{sol}
\end{problem}



% 4.13
\begin{problem}
Пусть $y_i = \beta_1 + \beta_2 x_{i2} + \beta_3 x_{i3} + \e_i$ — регрессионная модель, где $X = \begin{pmatrix} 1 & 0 & 0 \\ 1 & 0 & 0 \\ 1 & 0 & 0 \\ 1 & 1 & 0 \\ 1 & 1 & 1 \end{pmatrix}$, $y = \begin{pmatrix} 1 \\ 2 \\ 3 \\ 4 \\ 5 \end{pmatrix}$, $\beta = \begin{pmatrix} \beta_1 \\ \beta_2 \\ \beta_3 \end{pmatrix}$, $\e = \begin{pmatrix} \e_1 \\ \e_2 \\ \e_3 \\ \e_4 \\ \e_5  \end{pmatrix}$, ошибки $\e_i$ независимы и нормально распределены с $\E(\e)$ = 0, $Var(\e) = \sigma^2 I$. Для удобства расчётов даны матрицы: $X'X = \begin{pmatrix} 5 & 2 & 1 \\ 2 & 2 & 1\\ 1 & 1 & 1 \end{pmatrix}$ и $(X'X)^{-1}= \begin{pmatrix} 0.3333 & -0.3333 & 0.0000 \\ -0.3333 & 1.3333 & -1.0000 \\ 0.0000 & -1.0000 & 2.0000 \end{pmatrix}$


\begin{enumerate}
\item Укажите число наблюдений.
\item Укажите число регрессоров в модели, учитывая свободный член.
\item Найдите $TSS = \sum_{i=1}^n (y_i - \bar y)^2$.
\item Найдите $RSS = \sum_{i=1}^n (y_i - \hy_i)^2$.
\item Методом МНК найдите оценку для вектора неизвестных коэффициентов.
\item Чему равен $R^2$ в модели? Прокомментируйте полученное значение с точки зрения качества оценённого уравнения регрессии.\end{enumerate}
\begin{sol}

\begin{enumerate}
\item $n = 5$
\item $k = 3$
\item $TSS = 10$
\item $RSS = 2$
\item $\hb = \begin{pmatrix} \hb_1 \\ \hb_2 \\ \hb_3 \end{pmatrix} = (X'X)^{-1}X'y = \begin{pmatrix} 2 \\ 2 \\ 1 \end{pmatrix}$
\item $R^2 = 1 - \frac {RSS}{TSS} = 0.8.$ $R^2$ высокий, построенная эконометрическая модель хорошо описывает данные
\end{enumerate}
\end{sol}
\end{problem}



\begin{problem}

\begin{sol}
\end{sol}
\end{problem}




\section{Распределение}

\begin{problem}
Компоненты вектора $x=(x_1, x_2)'$ независимы и имеют стандартное нормальное распределение. Вектор $y$ задан формулой $y = (2x_1 + x_2 + 2, x_1 - x_2 - 1)$. 
\begin{enumerate}
  \item Выпишите совместную функцию плотности вектора $x$;
  \item Нарисуйте на плоскости линии уровня функции плотности вектора $x$;
  \item Выпишите совместную функцию плотности вектора $y$;
  \item Найдите собственные векторы и собственные числа ковариационной матрицы вектора $y$;
  \item Нарисуйте на плоскости линии уровня функции плотности вектора $y$. 
\end{enumerate}
\begin{sol}
\end{sol}
\end{problem}


\begin{problem}
Компоненты вектора $x=(x_1, x_2, x_3)'$ независимы и имеют стандартное нормальное распределение. 

\begin{enumerate}
\item Как выглядят в пространстве поверхности уровня совместной функции плотности?
\item Рассмотрим три апельсина с кожурой одинаковой очень маленькой толщины: бэби-апельсин радиуса $0.1$, стандартный апельсин радиуса $1$ и гранд-апельсин радиуса $10$. В кожуру какого апельсина вектор $x$ попадает с наибольшей вероятностью?
\item Мы проецируем случайный вектор на $x$ на плоскость $2x_1 + 3x_2 - 7x_3 = 0$. Какое распределение имеет квадрат длины проекции?
\item Введём вектор $y$ независимый от $x$ и имеющий такое же распределение. Спроецируем вектор $x$ на плоскость проходящую через начало координат и перпендикулярную вектору $y$.  Какое распределение имеет квадрат длины проекции? 
\end{enumerate}


\begin{sol}
Сферы с центром в начале координат. Проекция имеет хи-квадрат распределение с тремя степенями свободы.
Для нахождения максимальной вероятности максимизируем функцию
\[
\exp(-R^2/2) \cdot ((R+t)^3 - R^3) \to \max_R 
\],
где $R$ — радиус мякоти, а $t$ — толщина кожуры апельсина. Оставляем только линейную часть по $t$ и затем максимизируем.

Наибольшая вероятность попасть в апельсин радиуса $R=1$.
\end{sol}
\end{problem}



\begin{problem}
Пусть регрессионная модель $y_i = \beta_1 + \beta_2 x_{i2} + \beta_3 x_{i3} + \e_i$, $i = 1, \ldots, n$, задана в матричном виде при помощи уравнения $y = X \beta + \e$, где $\beta =  \begin{pmatrix}
\beta_1 & \beta_2 & \beta_3\\
\end{pmatrix} '$. Известно, что $\E \e = 0$ и $\Var (\e) = 4 \cdot I$. Известно также, что:

$y =  \begin{pmatrix}
1 \\
2 \\
3 \\
4 \\
5 \\
\end{pmatrix} $, $X =  \begin{pmatrix}
1 & 0 & 0 \\
1 & 0 & 0 \\
1 & 1 & 0 \\
1 & 1 & 0 \\
1 & 1 & 1 \\
\end{pmatrix} $

Для удобства расчётов ниже приведены матрицы:

$X' X =  \begin{pmatrix}
5 & 3 & 1 \\
3 & 3 & 1 \\
1 & 1 & 1 \\
\end{pmatrix} $ и $(X' X)^{-1} =  \begin{pmatrix}
0.5 & -0.5 & 0 \\
-0.5 & 1 & -0.5 \\
0 & -0.5 & 1.5 \\
\end{pmatrix} $.

Найдите:
\begin{enumerate}
\item $\Var (\e_1)$;
\item $\Var (\beta_1)$;
\item $\Var (\hb_1)$;
\item $\hVar(\hb_1)$;
\item $\E (\hb_1^2) - \beta_1^2$;
\item $\Cov (\hb_2, \hb_3)$;
\item $\hCov(\hb_2, \hb_3)$;
\item $\Var (\hb_2 - \hb_3)$;
\item $\hVar(\hb_2 - \hb_3)$;
\item $\Var (\beta_2 - \beta_3)$;
\item $\Corr (\hb_2, \hb_3)$;
\item $\hCorr(\hb_2, \hb_3)$;
\item $\E (\hs^2)$;
\item $\hs^2$.
\end{enumerate}


\begin{sol}
\begin{enumerate}
\item $\Var(\e_1)=\Var(\e)_{(1,1)}=4\cdot I_{(1,1)}=4$
\item $\Var(\beta_1)=0$, так как $\beta_1$ — детерминированная величина.
\item $\Var(\hb_1)=\sigma^2(X'X)^{-1}_{(1,1)}=0.5\sigma^2=0.5\cdot 4=2$
\item $\hVar(\hb_1)=\hat\sigma^2(X'X)^{-1}_{(1,1)}=0.5\hat\sigma^2_{(1,1)}=0.5\frac{RSS}{5-3}=0.25RSS=0.25y'(I-X(X'X)^{-1}X')y=0.25\cdot 1=0.25$

$\hat\sigma^2=\frac{RSS}{n-k}=\frac12$.

\item Так как оценки МНК являются несмещёнными, то $\E(\hb)=\beta$, значит:
\[
\E(\hb_1)-\beta_1^2=\E(\hb_1)-(\E(\hb_1))^2=\hVar(\hb_1)=0.25
\]

\item $\Cov(\hb_2,\hb_3)=\sigma^2(X'X)^{-1}_{(2,3)}=4\cdot\left(-\frac12\right)=-2$
\item $\hCov(\hb_2,\hb_3)=\hVar(\hat\beta)_{(2,3)}=\hat\sigma^2(X'X)^{-1}_{(2,3)}=\frac{1}{2}\cdot\left(-\frac12\right)=-\frac14$

\item $\Var(\hb_2-\hb_3)=\Var(\hb_2)+\Var(\hb_3)+2\Cov(\hb_2,\hb_3)=\sigma^2((X'X)^{-1}_{(2,2)}+(X'X)^{-1}_{(3,3)}+2(X'X)^{-1}_{(2,3)}=4(1+1.5+2\cdot(-0.5))=6$

\item $\hVar(\hb_2-\hb_3)=\hVar(\hb_2)+\hVar(\hb_3)+2\hCov(\hb_2,\hb_3)=\hat\sigma^2((X'X)^{-1}_{(2,2)}+(X'X)^{-1}_{(3,3)}+2(X'X)^{-1}_{(2,3)}=\frac{1}{2}\cdot1.5=0.75$

\item $\Var(\beta_2-\beta_3)=0$

\item $\corr(\hb_2,\hb_3)=\frac{\Cov(\hb_2,\hb_3)}{\sqrt{\Var(\hb_2)\Var(\hb_3)}}=\frac{-2}{\sqrt{4\cdot6}}=-\frac{\sqrt6}{6}$

\item $\hCorr(\beta_2,\beta_3)=\frac{\hCov(\hb_2,\hb_3)}{\sqrt{\hVar(\hb_2)\hVar(\hb_3)}}=\frac{-\frac14}{\sqrt{\frac12\cdot\frac34}}=-\frac{\sqrt6}{6}$

\item $(n-k)\frac{\hat\sigma^2}{\sigma^2}\sim\chi^2_{n-k}$.
\[
\E\left((n-k)\frac{\hat\sigma^2}{\sigma^2}\right)=n-k
\]
\[
\E\left(\frac{\hat\sigma^2}{2}\right)=1
\]
\[
\E(\hat\sigma^2)=2
\]

\item $\hat\sigma^2=\frac{RSS}{n-k}=\frac12$

\end{enumerate}

\end{sol}
\end{problem}


% \section{Хочу ещё задач!}




\Closesolutionfile{solution_file}


% для гиперссылок на условия
% http://tex.stackexchange.com/questions/45415
\renewenvironment{solution}[1]{%
         % add some glue
         \vskip .5cm plus 2cm minus 0.1cm%
         {\bfseries \hyperlink{problem:#1}{#1.}}%
}%
{%
}%

\section{Решения}
\protect \hypertarget {soln:1.1}{}
\begin{solution}{{1.1}}
\end{solution}
\protect \hypertarget {soln:1.2}{}
\begin{solution}{{1.2}}
Ответы: $0$, $0$, $0$, $\sum x_i^2$.
\end{solution}
\protect \hypertarget {soln:1.3}{}
\begin{solution}{{1.3}}
\begin{enumerate}
\item \(\htheta = \sum \left((y_i - z_i)(x_i - z_i) \right) / \sum \left(x_i - z_i\right)^2 \)
\end{enumerate}
\end{solution}
\protect \hypertarget {soln:1.4}{}
\begin{solution}{{1.4}}
\(\hat{\alpha} = 0, \ \hb = 1 \)
\end{solution}
\protect \hypertarget {soln:1.5}{}
\begin{solution}{{1.5}}
 % 1.5.
Рассмотрим регрессию суммы $(y_i + z_i)$ на саму себя. Естественно, в ней
\[
\widehat{y_i + z_i} = 0 + 1 \cdot (y_i + z_i).
\]

Отсюда получаем, что $\hat{\alpha} + \hat{\gamma} = 0$ и $\hb + \hat{\delta} = 1$.
\end{solution}
\protect \hypertarget {soln:1.6}{}
\begin{solution}{{1.6}}

Исходя из условия, нужно оценить методом МНК коэффициенты двух следующих моделей:
\[y_i = \alpha + \beta x_i + \e_i \]
\[y_i = \frac{\gamma}{2} + \frac{\delta}{2} x_i + \frac{1}{2} v_i \]

Заметим, что на минимизацию суммы квадратов остатков коэффициент \(1/2\) не влияет, следовательно:
\[\hat{\gamma} = 2\hat{\alpha}, \ \hat{\delta} = 2 \hb  \]

\end{solution}
\protect \hypertarget {soln:1.7}{}
\begin{solution}{{1.7}}
Выпишем задачу:
\[
\begin{cases}
RSS = \sum\limits_{i=1}^{n}(y_i - \hb_1x_i - \hb_2z_i)^2 \rightarrow \min\limits_{\hb_1, \hb_2}\\
\hb_1 + \hb_2 = 1
\end{cases}
\]

Можем превратить ее в задачу минимизации функции одного аргумента:
\[
RSS =  \sum\limits_{i=1}^{n}(y_i - x_i - \hb_2(z_i-x_i))^2 \rightarrow \min_{\hb_2}
\]

Выпишем условия первого порядка:
\[
\frac{\partial RSS}{\partial \hb_2} = \sum\limits_{i=1}^{n}2(y_i-x_i-\hb_2(z_i-x_i))(x_i-z_i)=0
\]

Отсюда:
\[
\sum\limits_{i=1}^{n}(y_i-x_i)(x_i-z_i) + \hb_2\sum\limits_{i=1}^{n}(z_i-x_i)^2 = 0 \Rightarrow \hb_2 = \frac{\sum\limits_{i=1}^n (y_i-x_i)(z_i-x_i)}{\sum\limits_{i=1}^n (z_i-x_i)^2}
\]

А $\hb_1$ найдется из соотношения $\hb_1+\hb_2 = 1$.

\end{solution}
\protect \hypertarget {soln:1.8}{}
\begin{solution}{{1.8}}
Обозначив вес первого слитка за \(\beta_1\), вес второго слитка за \(\beta_2\), а показания весов за \(y_i\), получим, что
\[y_1 = \beta_1 + \e_1, \ y_2 = \beta_2 + \e_2, \ y_3 = \beta_1 + \beta_2 + \e_3\]

Тогда
\[(300 - \beta_1)^2 + (200 - \beta_2)^2 + (400 - \beta_1 - \beta_2)^2 \rightarrow \min \limits_{\beta_1,\  \beta_2} \]
\[\hb_1 = \frac{800}{3}, \ \hb_2 = \frac{500}{3} \]
\end{solution}
\protect \hypertarget {soln:1.9}{}
\begin{solution}{{1.9}}
Можем воспользоваться готовой формулой для регрессии на константу:
\[
\hb = \bar{y} = \frac{10+10+3}{3} = \frac{23}{3}
\]

(можно решить задачу $2(10-\beta)^2 + (3-\beta)^2\rightarrow \min$)

\end{solution}
\protect \hypertarget {soln:1.10}{}
\begin{solution}{{1.10}}
\end{solution}
\protect \hypertarget {soln:1.11}{}
\begin{solution}{{1.11}}
\end{solution}
\protect \hypertarget {soln:1.12}{}
\begin{solution}{{1.12}}
\end{solution}
\protect \hypertarget {soln:1.13}{}
\begin{solution}{{1.13}}
Пусть \(\bar{y}\) — средний \(y\) до добавления нового наблюдения, \(\bar{y}'\) — после добавления нового наблюдения. Будем считать, что изначально было \(n\) наблюдений. Заметим, что
\[\bar{y}' = \frac{(y_1 + \ldots + y_n) + y_{n+1}}{n + 1} = \frac{n \bar{y} + y_{n + 1}}{n + 1} = \frac{n}{n+ 1}\bar{y} + \frac{1}{n+1}y_{n+1}\]

Покажем, что \(TSS\) может только увеличится при добавлении нового наблюдения (остается неизменным при \(y_{n+1} = \bar{y}\)):
\[TSS'= \sum_{i = 1}^{n + 1} (y_i - \bar{y}')^2 = \sum_{i = 1}^{n} (y_i - \bar{y} + \bar{y} - \bar{y}')^2 + (y_{n + 1} - \bar{y}')^2 = \]
\[=\sum_{i = 1}^{n} (y_i - \bar{y})^2 + n(\bar{y} - \bar{y}')^2 + (y_{n + 1} - \bar{y}')^2  = TSS + \frac{n}{n+1} (y_{n+1} - \bar{y})^2\]

Следовательно, \(TSS' \geqslant TSS\).

Также сумма \(RSS\) может только вырасти или остаться постоянной при добавлении нового наблюдения. Действительно, новое $(n+1)$-ое слагаемое в сумме неотрицательно. А сумма $n$ слагаемых минимальна при старых коэффициентах, а не при новых.

\(ESS\) и \(R^2\) могут меняться в обе стороны. Например, рассмотрим ситуацию, где точки лежат симметрично относительно некоторой горизонтальной прямой. При этом $ESS=0$. Добавим наблюдение — $ESS$ вырастет, удалим наблюдение — $ESS$ вырастет.
\end{solution}
\protect \hypertarget {soln:1.14}{}
\begin{solution}{{1.14}}
\begin{enumerate}
\item $R^2$ упал до нуля.
\item Да, можно. Если добавить точку далеко слева внизу от исходного набора данных, то наклон линии регрессии будет положительный. Если далеко справа внизу, то отрицательный. Будем двигать точку так, чтобы поймать нулевой наклон прямой. Получим $ESS=0$.
\end{enumerate}
\end{solution}
\protect \hypertarget {soln:2.15}{}
\begin{solution}{{2.15}}
\end{solution}
\protect \hypertarget {soln:2.16}{}
\begin{solution}{{2.16}}
\end{solution}
\protect \hypertarget {soln:2.17}{}
\begin{solution}{{2.17}}
\end{solution}
\protect \hypertarget {soln:2.18}{}
\begin{solution}{{2.18}}
\end{solution}
\protect \hypertarget {soln:3.19}{}
\begin{solution}{{3.19}}
\end{solution}
\protect \hypertarget {soln:3.20}{}
\begin{solution}{{3.20}}
Сферы с центром в начале координат. Проекция имеет хи-квадрат распределение с тремя степенями свободы.
Для нахождения максимальной вероятности максимизируем функцию
\[
\exp(-R^2/2) \cdot ((R+t)^3 - R^3) \to \max_R
\],
где $R$ — радиус мякоти, а $t$ — толщина кожуры апельсина. Оставляем только линейную часть по $t$ и затем максимизируем.

Наибольшая вероятность попасть в апельсин радиуса $R=1$.
\end{solution}


\section{Источники мудрости}
\printbibliography[heading=none]


\end{document}
