\protect \hypertarget {soln:1.1}{}
\begin{solution}{{1.1}}
\end{solution}
\protect \hypertarget {soln:1.2}{}
\begin{solution}{{1.2}}
\end{solution}
\protect \hypertarget {soln:1.3}{}
\begin{solution}{{1.3}}
\[
y_n - \hat y_n = (1 - H_{nn}) (y_n - \hat y_n^-)
\]
\end{solution}
\protect \hypertarget {soln:2.4}{}
\begin{solution}{{2.4}}
$(5, 6, -7)$
\end{solution}
\protect \hypertarget {soln:2.5}{}
\begin{solution}{{2.5}}
\end{solution}
\protect \hypertarget {soln:2.6}{}
\begin{solution}{{2.6}}
\end{solution}
\protect \hypertarget {soln:2.7}{}
\begin{solution}{{2.7}}
\end{solution}
\protect \hypertarget {soln:2.8}{}
\begin{solution}{{2.8}}
\end{solution}
\protect \hypertarget {soln:2.9}{}
\begin{solution}{{2.9}}
\end{solution}
\protect \hypertarget {soln:2.10}{}
\begin{solution}{{2.10}}
\end{solution}
\protect \hypertarget {soln:2.11}{}
\begin{solution}{{2.11}}
\end{solution}
\protect \hypertarget {soln:2.12}{}
\begin{solution}{{2.12}}
\end{solution}
\protect \hypertarget {soln:2.13}{}
\begin{solution}{{2.13}}
\end{solution}
\protect \hypertarget {soln:2.14}{}
\begin{solution}{{2.14}}
\end{solution}
\protect \hypertarget {soln:2.15}{}
\begin{solution}{{2.15}}
  Нет. Не выполнено $\tilde{L} \geq L$ для всех $M \in \RR$.
\end{solution}
\protect \hypertarget {soln:2.16}{}
\begin{solution}{{2.16}}
\end{solution}
\protect \hypertarget {soln:2.17}{}
\begin{solution}{{2.17}}
\end{solution}
\protect \hypertarget {soln:2.18}{}
\begin{solution}{{2.18}}
\end{solution}
\protect \hypertarget {soln:3.19}{}
\begin{solution}{{3.19}}
\end{solution}
\protect \hypertarget {soln:3.20}{}
\begin{solution}{{3.20}}
Сферы с центром в начале координат. Проекция имеет хи-квадрат распределение с тремя степенями свободы.
Для нахождения максимальной вероятности максимизируем функцию
\[
\exp(-R^2/2) \cdot ((R+t)^3 - R^3) \to \max_R
\],
где $R$ — радиус мякоти, а $t$ — толщина кожуры апельсина. Оставляем только линейную часть по $t$ и затем максимизируем.

Наибольшая вероятность попасть в апельсин радиуса $R=1$.
\end{solution}
